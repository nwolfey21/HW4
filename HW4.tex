\documentclass[12pt]{article}

\usepackage{amsmath,amsfonts,amsthm,amssymb}
\usepackage{color}
\usepackage{graphicx}
\usepackage{wrapfig}
\usepackage{epsfig}
%\usepackage{subfigure}
\usepackage{times}
\usepackage{xspace}
\usepackage{url}
\usepackage{pdfpages}
\usepackage{array}
\usepackage{subfig}
\usepackage{multirow}
\usepackage{dblfloatfix} 

\setlength{\topmargin}{0.0in}     % top of paper to head (less one inch)
\setlength{\headheight}{0in}      % height of the head
\setlength{\headsep}{0in}         % head to the top of the body
\setlength{\textheight}{9.0in}    % height of the body
\setlength{\oddsidemargin}{-.25in} % left edge of paper to body (less one inch)
\setlength{\evensidemargin}{0mm}  % ditto, even pages
\setlength{\textwidth}{7.0in}     % width of body
\setlength{\topskip}{0in}         % top of body to bottom of first line of text
\setlength{\parindent}{1pc}       

\begin{document}
\thispagestyle{empty}
%%%%%%%%%%%%%%%%%%%%%%%%%%%%%%%%%%%%%%%%%%%%%%%%%%%%%%%%%%%%%%%%%%%%%%%%%%%%%%%%%%%%%%%%
% Project Summary
%%%%%%%%%%%%%%%%%%%%%%%%%%%%%%%%%%%%%%%%%%%%%%%%%%%%%%%%%%%%%%%%%%%%%%%%%%%%%%%%%%%%%%%%     vvvvvvvvvvvv
\begin{center}
{\Large\bf Assignment \#4: Data Collection and Preparation}
\vspace{3mm}
\\Caitlin Ross \& Noah Wolfe
\\02/18/2016
\\*[3mm]
\end{center}
%%%%%%%%%%%%%%%%%%%%%%%%%%%%%%%%%%%%%%%%%%%%%%%%%%%%%%%%%%%%%%%%%%%%%%%%%%%%%%%%%%%%%%%%
% Task Objects
%%%%%%%%%%%%%%%%%%%%%%%%%%%%%%%%%%%%%%%%%%%%%%%%%%%%%%%%%%%%%%%%%%%%%%%%%%%%%%%%%%%%%%%%
\section{Data Source (ROSS)}
%Description of the data source and corresponding applications.
In this assignment, we have decided to generate/collect data from our ROSS research projects. 
ROSS is a massively parallel discrete-event simulator that
can process billions of events per second \cite{Holder}, \cite{Bauer}. ROSS
models are made up of a collection of logical processes (LPs).
Each LP models a distinct component of the system. LPs
interact with one another through events in the form of time-stamped messages. An MPI task is abstracted as a processor
element (PE) in ROSS. Each PE owns a number of LPs and
schedules events in time-stamp order for all LPs assigned to
it. Events that are destined for a logical process on another
PE (i.e. remote events) are sent as MPI messages. 

The underlying ROSS simulator is itself, a very complex piece of software with a large set of parameters. When changed, these parameters can trigger large changes in metrics such as number of reverse computations, remote events, event efficiency, etc. The only data collection is averaged over all processes and collected at the very end of the simulation. 

\section{Research Questions}
%Write down at least 2 specific research questions that can be solved by analyzing this data. The first should be "obvious" and may simply communicate the overall quantity of data you've got your hands on. The second should be more complex or subtle, that can be answered by the data, but will involve rearranging or simplifying or finding correlations within the data.

ROSS is designed to be a very fast and efficient simulator. In order to get the best possible speed-up, developers need to know what is going on behind the curtains. We need to know not only if a simulation is experiencing hotspots, but also be able to detect the cause of those hotspots. 

\section{Hypotheses}
%What are your specific hypotheses related to these research questions? What knowledge are your drawing on to make these predictions?

\section{Data Format}
%With your research questions in mind, design the detailed format for your raw data (the columns of your data "spreadsheet") and decide on the action or sampling frequency for each "row" of the data. Make sure you are able to acquire an "interesting" amount of data, both number of samples (at least 1000 rows?) and dimensions per sample (at least 3 columns?) Note: These estimates are not requirements. If your data has many more columns, things can be quite interesting even with far fewer rows.

\section{Data Extraction \& Manipulation}
%Detail the efforts you made to collect, parse, reorganize, simplify, and/or post-process this data source.

\section{Sample Visualizations}
%Create (at least 2) simple visualization plots of this data using a tool that's new to you (or you would like to learn more about). Consider using: Excel, LineUp, Tableau, Google Analytics, Plotly, or VTK. These plots should attempt to answer the research questions you posed earlier. You can revise your research questions as needed as you work with the data.

\section{Visualization Tool Review}
%A brief review of the tool you used to create the visualizations.

% Bibliography
\bibliographystyle{abbrv}
\bibliography{HW4}

\end{document}


%\begin{figure}[!ht]
%     \centering
%     \subfloat[][twopi]{\epsfig{file=figures/MMS7-3-17.png, height=2.2in, width=4.80in}\label{vis-100-1}}\\
%     \subfloat[][dot]{\epsfig{file=figures/MMS7-3-12.png, height=2.0in, width=3.00in}\label{vis-100-2}}
%     \subfloat[][circo]{\epsfig{file=figures/MMS7-3-circo.png, height=2.0in, width=3.0in}\label{vis-100-3}}\\
%     \caption{Visualizations of the class social network data using a collection of Graphviz graph drawing programs. Red lines represent "before RPI" connections, blue represent "lived with" connections and black represent "Data Structures" connections. }
%     \label{vc-occupancy}
%\end{figure}

%\begin{figure}[!h]
%\centering
%\epsfig{file=figures/storyboard.jpg, width=5.5in}
%\caption{Storyboard showing the desired layout and interaction of the network visualization tool in D3.}
%\label{good}
%\end{figure}